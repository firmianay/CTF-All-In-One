\chapter{前言}\index{前言}

\begin{quote}
    “与其相信谣言,不如一起学习”。
\end{quote}

\section{常见问题}

\begin{itemize}\setlength{\parindent}{2em}
    \item 什么是 CTF,我为什么要学?
    
    CTF 是网络安全技术人员之间进行技术竞技的一种比赛形式,通过学习,可以在法律允许的范围内,快速地了解和掌握相关安全技术。

    \item 阅读本书的预备知识是什么?
    
    本书是为初学者准备的,不要求有预备知识,但如果对 Linux 操作系统,对编程有一定了解,肯定会有帮助。

    \item 我可以买到纸质版吗?
    
    抱歉,目前没有。除非对纸质书有偏好,你可以自行打印。否则由于内容整体尚未完成,且更新很快,作者更推荐使用电子版。
    
    \item 本书有PDF/epub/mobi格式的吗?
    
    目前没有 epub/mobi 版本。暂时有 pdf,可在 GitBook 页面下载,这群人正在努力学习 \LaTeX\␣ 的使用,以提供更优雅的排版和PDF文件。
    
    \item 我能打印本书或者作为教材教课吗?
    
    太棒了!必须的!本书使用 Creative Commons license (CC BY-SA 4.0),大可随意使用。作为一个开源项目,我们当然希望更多的人了解并参与进来。
    
    \item 本书为何免费,有何目的?
    
    技术类书籍大多赚不到钱,只是作者的兴趣使然,顺便给自己打个广告。让更多的人了解并推广,才是我们的目的所在,也是开源精神之所在。
    
    \item 我还有其他问题。
    
    你也可以选择提交issue到本仓库,也可以邮件联系我们。
    
\end{itemize}

\section{合作与贡献}
\indent \setlength{\parindent}{2em}

    \indent 随着信息安全的迅速发展,CTF 竞赛也在如火如荼的开展,有人说 “今天的 ACM 就是明天的 CTF” ,颇有几分道理。
    \\目前市场上已经充斥着大量的 ACM 书籍,而 CTF 以其知识内容之分散、考察面之广泛、题目类型之多变,让许多新手不知所措,同时也加大了该方面书籍的编写难度。

    \indent 此书本着开源之精神,以分享他人提高自己为目的,将是一本大而全的 CTF 领域指南。因本人能力和时间有限,不可能精通各个类别的知识,欢迎任何人提出任何建议,和我一起完成此书。千万不要觉得自己是初学者就不敢提交 PR(issue),千万不要担心自己提交的 PR(issue) 会有问题,毕竟最后合并的人是我,背锅的也是我:)

    \indent 如果你还有关于本书的其他想法,请直接给我发邮件 firmianay@gmail.com。

\section{致谢}
\indent \setlength{\parindent}{2em}

\indent 感谢内容贡献者:skyel1u、phantom0301

\indent 感谢 XDSEC,把我引上了安全这条路,认识了很多志同道合的小伙伴。

\indent 感谢 GitHub 上的朋友,是你们的 star 给我写作的动力。
\chapter{前言}\index{前言}

\begin{quote}
    “与其相信谣言,不如一起学习”。
\end{quote}

\section{常见问题}

\begin{itemize}\setlength{\parindent}{2em}
    \item 什么是 CTF,我为什么要学?
    
    CTF 是网络安全技术人员之间进行技术竞技的一种比赛形式,通过学习,可以在法律允许的范围内,快速地了解和掌握相关安全技术。

    \item 阅读本书的预备知识是什么?
    
    本书是为初学者准备的,不要求有预备知识,但如果对 Linux 操作系统,对编程有一定了解,肯定会有帮助。

    \item 我可以买到纸质版吗?
    
    抱歉,目前没有。除非对纸质书有偏好,你可以自行打印。否则由于内容整体尚未完成,且更新很快,作者更推荐使用电子版。
    
    \item 本书有PDF/epub/mobi格式的吗?
    
    目前没有 epub/mobi 版本。暂时有 pdf,可在 GitBook 页面下载,这群人正在努力学习 \LaTeX\␣ 的使用,以提供更优雅的排版和PDF文件。
    
    \item 我能打印本书或者作为教材教课吗?
    
    太棒了!必须的!本书使用 Creative Commons license (CC BY-SA 4.0),大可随意使用。作为一个开源项目,我们当然希望更多的人了解并参与进来。
    
    \item 本书为何免费,有何目的?
    
    技术类书籍大多赚不到钱,只是作者的兴趣使然,顺便给自己打个广告。让更多的人了解并推广,才是我们的目的所在,也是开源精神之所在。
    
    \item 我还有其他问题。
    
    你也可以选择提交issue到本仓库,也可以邮件联系我们。
    
\end{itemize}

\section{合作与贡献}
\indent \setlength{\parindent}{2em}

    \indent 随着信息安全的迅速发展,CTF 竞赛也在如火如荼的开展,有人说 “今天的 ACM 就是明天的 CTF” ,颇有几分道理。
    \\目前市场上已经充斥着大量的 ACM 书籍,而 CTF 以其知识内容之分散、考察面之广泛、题目类型之多变,让许多新手不知所措,同时也加大了该方面书籍的编写难度。

    \indent 此书本着开源之精神,以分享他人提高自己为目的,将是一本大而全的 CTF 领域指南。因本人能力和时间有限,不可能精通各个类别的知识,欢迎任何人提出任何建议,和我一起完成此书。千万不要觉得自己是初学者就不敢提交 PR(issue),千万不要担心自己提交的 PR(issue) 会有问题,毕竟最后合并的人是我,背锅的也是我:)

    \indent 如果你还有关于本书的其他想法,请直接给我发邮件 firmianay@gmail.com。

\section{致谢}
\indent \setlength{\parindent}{2em}

\indent 感谢内容贡献者:skyel1u、phantom0301

\indent 感谢 XDSEC,把我引上了安全这条路,认识了很多志同道合的小伙伴。

\indent 感谢 GitHub 上的朋友,是你们的 star 给我写作的动力。
